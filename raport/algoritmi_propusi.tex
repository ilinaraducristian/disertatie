\section{Algoritmi de implementare propusi}
\subsection{NSGA-II și NSGA-III}
\subsubsection{NSGA-II}
\paragraph{}
Principalele probleme ale algoritmului genetic de sortare nedominată (NSGA) sunt: complexitate de calcul ridicată pentru sortarea nedominată, lipsa elitismului care s-a dovedit foarte importantă în performanțele unui algoritm genetic și necesitatea unui parametru comun $\sigma_{share}$. \cite{nsgaii}
\paragraph{}
Algoritmul NSGA îmbunătățit (NSGA-II) propune soluții pentru probleme descrise și s-a dovedit a fi mai bun decât alți doi algoritmi similari PAES și SPEA.
\paragraph{}
Pentru a identifica soluțiile din primul front nedominat dintr-o populație N, fiecare soluție poate fi comparată cu fiecare soluție din populație pentru a afla dacă este dominată. Pentru asta avem nevoie de O(MN) comarații pentru fiecare soluție, unde M este numărul de obiective. \cite{nsgaii}
\subsubsection{NSGA-III}
\paragraph{}
În cele mai multe cazuri, problemele de optimizare multiobiective sunt definite ca avand cel puțin 4 obiective, iar cele cu mai puțin de 4 obiective sunt tratate ca facând parte din altă clasă de algoritmi deoarece frontul Pareto generat poate fi vizualizat cu ușurință. De asemenea, numărul de obiective maxim des întâlnit este curpins intre 10 si 15. Dificutățile întâmpinate de algoritmii evolutivi multiobiectivi ce folosesc principiul de dominare sunt: o proporție mare din populație își pierde dominanța, complexitatea de calcul crește semnificativ cu numărul de obiective, recombinarea ineficientă a soluțiilor, colectarea de metrice crește costul complexității de calcul și datorită necesității unei populații mai mari, vizualizarea este mult mai greoaie. \cite{nsgaiii}
\paragraph{}
Algoritmul propus (NSGA-III) diferă de algoritmul original (NSGA-II) prin modificări semnificative ale operatorului de selecție și menținerea diversității populației este ajutată prin aplicarea adaptivă a unui număr de puncte de referință. \cite{nsgaiii}
\subsection{Algoritmi de tip ACOR}
Algoritmul euristic de optimizare pe baza coloniei de furnici funcționează după o paradigmă numită sistem de furnici este propus ca o nouă abordare pentru optimizarea combinatorică stocastică. Principalele caracteristici ale acestui model sunt: raspunsul pozitiv contribuie la descoperirea rapidă a soluțiilor bune, calculul distribuit evită o convergență prematură și folosirea unui euristic constructiv lacom ajută la găsirea unor soluții acceptabile la începutul procesului de căutare.\cite{aco}
\subsubsection{SOACOR}
\subsubsection{MOACOR}