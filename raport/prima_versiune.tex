\subsection{Prima versiune}
\paragraph{}
Algoritmul multiobiectiv definește următorii parametrii cu valori predefinite:
\begin{itemize}
    \item localitateaProcesuluiDeCautare 0.1
    \item vitezaDeConvergenta 0.85
    \item numarFurnici 100
    \item selectieTournamet false
\end{itemize}
\paragraph{}
O iterație a algoritmului presupune calcularea ponderilor și a probabilităților de selecție, construcția și evaluarea soluțiilor, calcularea furnicii dominante și reducerea populației la numărul de furnici definit la creare algoritmului prin metoda de sortare nedominată.
\paragraph{}
Un element din vectorulu de ponderi ce are numărul de elemente egal cu mărimea populației, este definit astfel:
\begin{equation}
    \epsilon^{\frac{-1*i^2}{2*constanta^2*marimea\ populatiei^2}}*\frac{1}{constanta}
\end{equation}
unde:
\begin{itemize}
    \item $constanta = localitatea\ procesului\ de\ cautare * marimea\ populatiei * \sqrt{2\pi}$
\end{itemize}
\paragraph{}
Probabilitățile de selcție reprezintă un vector format din ponderile calculate antecedent unde fiecare element este împărțit la suma ponderilor.
Construcția souțiilor se face în funcție de parametrul algorimului "selectieTournamet". În funcție de acesta, se va obține un indice ce va fi folosit în calcularea deviațiilor standard și pentru a construi o nouă soluție. Deviațiile standard se obțin cu ajutorul următoarei formule:
\begin{equation}
    deviatie\ standard_i = \frac {viteza\ de\ convergenta * (\sum_{i=0}^{n} variabila\ solutiei_i - variabila\ solutiei_indice)} {marimea\ populatiei - 1}
\end{equation}
unde:
\begin{itemize}
    \item n = mărimea populației
    \item indice = obținut prin selecție tournament sau prin extragerea unui număr aleator din vectorul probabilităților cumulate de selcție
\end{itemize}
\paragraph{}
Construcția unei noi soluții se face prin alegerea unui individ din populație cu ajutorul indicelui calculat antecedent și ajustarea variabilelor acestuia cu vectorul deviațiilor standard.
\paragraph{}
O încercare de a îmbunătății acestă versiune a fost obținută prin modificarea formulei de calcul a ponderilor utilizate în deviațiile standard:
\begin{equation}
    \epsilon^{\frac{-1*i^2}{2*localitatea\ procesului\ de\ cautare^2*marimea\ populatiei^2}}*\frac{1}{constanta}
\end{equation}
